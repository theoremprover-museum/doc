\documentclass[12pt]{article}
\usepackage{url}
\usepackage[show]{ed}
\usepackage[hyperref,backend=bibtex,style=alphabetic]{biblatex}
\addbibresource{kwarcpubs.bib}
\addbibresource{extpubs.bib}
\addbibresource{kwarccrossrefs.bib}
\addbibresource{extcrossrefs.bib}
\usepackage{bibtweaks}

\title{The Theorem Prover Museum\\ Conserving the System Heritage of Automated Reasoning}
\author{Michael Kohlhase\\Computer Science, FAU Erlangen-N\"urnberg\\\url{http://kwarc.info/kohlhase}}
\begin{document}
\maketitle
\begin{abstract}
  \ednote{tbw}
\end{abstract}
\section{Introduction}\label{sec:intro}

Theorem provers are software systems that can find or check proofs for conjectures given
in some logic. Research in theorem proving system started with the logical theorist 1955
and has led to a succession of systems since.

Theorem provers are complex software systems that have pushed the envelope of artificial
intelligence and programming, and as such they constitute important cultural artefacts.

With the current wave of retirements of the original principal investigators there is good
chance that the systems are lost, when their group servers are shut down. This web site
aims to preserve the ones we can still get our hands on. This idea is compatible with the
Software Heritage initiative~\cite{SoftwareHeritage:on}, and contributes since it is based
on GitHub repositories.

The term ``museum'' may be sound bit ambitious, since the exhibition and didactic
interpretation of the theorem provers is beyond our scope (and perhaps abilities). But the
foremost function of a museum is the conservation of artefacts, which is what the "theorem
prover museum" project intends to do.

Note that it is not the purpose of this site to keep the theorem proving systems running
(in many cases the compilers and dependencies have moved on, making this very
difficult). But only to archive the source code for academic study. In particular this
should lower the barrier of archiving systems here.

\section{The Museum}

The actual ``theorem prover museum'' consists of a simple web site that features a list of
GitHub repositories that contain the actual source code 

This site is the front-end to a collection of source code repositories for theorem provers
(see below). To exhibit your system here or suggest a system for inclusion (most wanted
list, systems believed lost), please contact Michael Kohlhase. Contributors,
community/contact, project/issues.

\section{Call for Contributors}

\section{Conclusion}\label{sec:concl}
\printbibliography
\end{document}
%%% Local Variables:
%%% mode: latex
%%% TeX-master: t
%%% End:
