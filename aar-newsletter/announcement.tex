\documentclass{article}
\usepackage{url,a4wide}
\title{A Museum for Theorem Provers\\ and Archive for their Source Code}
\author{Michael Kohlhase\\FAU Erlangen N\"urnberg\\\url{http://kwarc.info/kohlhase}}
\begin{document}
\maketitle

Theorem provers are complex software systems that have pushed the envelope of artificial
intelligence and programming, and as such they constitute important cultural artefacts.
With the ongoing wave of retirements of the original principal investigators there is a
good chance that these systems are lost, when their servers are shut down. The following
incident is unfortunately quite typical.

When -- ten days after Herbert Simon's passing in
February 2001 -- the author tried to find a copy of the source code of the Logic Theorist
in Simon's scientific estate at CMU, all tapes and printouts had already been discarded --
only the written materials and notes were being catalogued in the CMU
library. Fortunately, report P-868 of the Rand Corporation, where the program was
conceived, contains the full printout of the code. Otherwise we would only be able to read
about this seminal program, but not be able to study the artefact itself.

In other cases, we may not have been so lucky, this is a great loss to the scientific
community who may want to find out just how new techniques came about in automated
reasoning.  

The ``Theorem Prover Museum'' (\url{http://theoremprover-museum.github.io}) is an
initiative to conserve the source code of theorem provers. It is a collection of GitHub
repositories with source code of these systems, together with a web site that presents
them and organizes the process of acquiring more: In the 18 months since the initiative
was started, the museum has gained the source code of 28 theorem provers.
 
Note that it is not the purpose of the museum to keep the theorem proving systems running
(in many cases the compilers and dependencies have moved on, making this very
difficult). But only to archive the source code for academic study.  This is a
well-considered design decision, taken to lower the barrier of archiving systems
here. Again, once the source code is preserved -- i.e. made public by the original authors
-- other enthusiasts can revive it. Indeed this has already happened, triggered by the act
of exposing the source.

Another unexpected benefit from exposing the systems is that the \textsf{swMATH}
information system for mathematical software (\url{http://swmath.org}) lists the museum as
one of its special categories. This links systems to their traces in the mathematical
literature and aggregates information from other sources.

But to be successful, the theorem prover museum must be a community effort, so think about
whether you have access to historic (or current) source code or can talk to colleagues who
do. Suggestions, bug reports, etc, are best communicated as GitHub issues at
\url{https://github.com/theoremprover-museum/theoremprover-museum.github.io/issues}.

\end{document}

%%% Local Variables:
%%% mode: latex
%%% TeX-master: t
%%% End:

%  LocalWords:  maketitle swMath
