\documentclass[smallcondensed]{svjour3}  
\usepackage{url}
\usepackage[show]{ed}
\usepackage[hyperref,backend=bibtex,style=alphabetic]{biblatex}
\addbibresource{kwarcpubs.bib}
\addbibresource{extpubs.bib}
\addbibresource{kwarccrossrefs.bib}
\addbibresource{extcrossrefs.bib}
\usepackage{bibtweaks}
\usepackage{hyperref}

\title{The Theorem Prover Museum}
\subtitle{Conserving the System Heritage of Automated Reasoning}
\author{Michael Kohlhase}
\institute{Computer Science, FAU Erlangen-N\"urnberg\\\url{http://kwarc.info/kohlhase}}
\journalname{Journal of Automated Reasoning}
\begin{document}
\maketitle
\begin{abstract}
  \ednote{tbw}
  \keywords{theorem provers \and museum \and source code \and conservation}
\end{abstract}
\section{Introduction}\label{sec:intro}

Theorem provers are software systems that can find or check proofs for conjectures given
in some logic. Research in theorem proving systems started with Newell and Simon's ``logic
theorist'' 1955~\cite{NewSim:ltmcips56} -- one of the earliest systems in the
then-emerging field of Artificial Intelligence -- and has led to a succession of systems
since. Today, more than 60 years later, the CADE ATP system competition~\cite{CASC}
attracts 15-20 systems annually. Automated reasoning systems have applications ranging
from the verification of mathematical results, via program synthesis/verification, the
Semantic Web, all the way to the discovery of unfair trading rules in darkpools of
investment banks.

Theorem provers are complex software systems that have pushed the envelope of artificial
intelligence and programming, and as such they constitute important cultural
artefacts.\ednote{give some examples, some things that were first done in theorem provers,
  e.g. the programming language ML which heavily influenced modern typed functional
  programs was introduced as a meta-language of the LCF theorem prover by Robin
  Milner. Its type system was motivated by the idea that proofs could be programmed, if
  the type of proofs can only contain logically valid proofs.}

With the ongoing wave of retirements of the original principal investigators there is good
chance that these systems are lost, when their group servers are shut down. The following
incident is unfortunately quite typical. When -- ten days after Herbert Simon's passing in
February 2001 -- the author tried to find a copy of the source code of the Logic Theorist
in Simon's scientific estate at CMU, all tapes and printouts had already been discarded --
only the written materials and notes were being catalogued in the CMU
library. Fortunately, report P-868 of the Rand Corporation~\cite{NewSim:ltmcips56}, where
the program was conceived contains the full printout of the code. Otherwise we would only
be able to read about this seminal program, but not be able to study the artefact
itself. 

In other cases, we may not have been so lucky; see~\cite{tpmuseum:tpbl:on} for a list of
theorem provers believed lost. This is a great loss to\ednote{explain why!, history of
  science/engineering.}

\section{A Museum of Theorem Prover Source Code}

This article reports on an initiative started by the author in spring 2016 to help
conserve the source code of theorem provers: the ``theorem prover museum'', a collection
of GitHub repositories repositories with source code of systems, together with a web site
that presents them and organizes the process of acquiring more.

The term ``museum'' in the title may be sound bit ambitious, since the exhibition and
didactic interpretation of the theorem provers is beyond the scope of the initiative (and
perhaps abilities of the founder). But the foremost function of any museum is the
conservation of artefacts, which is what the ``theorem prover museum'' project intends to
do. Once the source code is preserved, historians of science and engineering can start to
do research on it and create multiple user interfaces to present it to the public. 

Note that it is not the purpose of the museum to keep the theorem proving systems running
(in many cases the compilers and dependencies have moved on, making this very
difficult). But only to archive the source code for academic study.  This is a
well-considered design decision, taken to lower the barrier of archiving systems
here. Again, once the source code is preserved -- i.e. made public by the original authors
-- other enthusiasts can revive it. Indeed this has already happened, triggered by the act
of exposing the source.

\section{The Museum}

The actual ``theorem prover museum'' consists of a simple web site that features a list of
GitHub repositories that contain the actual source code. The repositories are collected in
the GitHub organisation \texttt{theoremprover-museum}
\url{https://github.com/theoremprover-museum} and the web site consists of a central index
page at \url{https://theoremprover-museum.github.io/}, and various administrative pages
that collect systems, e.g. a list of ``most wanted systems'', a list of ``theorem provers
believed lost''~\cite{tpmuseum:tpbl:on}, and a list of ``active systems''. Once in a
while, a request for the source code of a system that has fallen below the radar of the
community is met with an exasperated reply like ``but Ontic lives!!!'' (David McAllister
in 2016) -- this the last page.

In the 18 months since the initiative was started, the museum has gained the source code
of 28 systems\ednote{update}.

The concept of the theorem prover museum is compatible with the Software Heritage
Initiative~\cite{SoftwareHeritage:on}, and particular GitHub-based implementation
contributes to it automatically, since the SHI indexes GitHub repositories and the museum
increases these.

The \textsf{swMath} information system for mathematical software~\cite{swMath:on} lists
the museum as one of its special categories~\cite{swMath:tpmuseum:on}. This links systems
to their traces in the mathematical literature -- unfortunately, much of the theorem
proving literature is in Computer Science conferences, which are only partially tracked in
the unterlying \textsf{zbMATH} abstracting service~\cite{zbMATH:on}, but CS does not have
a comparable system. Even so, 

\section{Related Initiatives and Resources}
We list other public resources that may give further information
\begin{itemize}
\item the Encyclopedia of Proof Systems~\cite{Wolzenlogel-Paleo:teps17} collects proof
  systems that are mechanized by the theorem provers.
\item the Wikipedia page on automated theorem provers contains a list of systems
\end{itemize}

\section{Call for Contributions}

To exhibit your system here or suggest a system for inclusion (most wanted
list, systems believed lost), please contact Michael Kohlhase. Contributors,
community/contact, project/issues.

\section{Conclusion}\label{sec:concl}

\begin{acknowledgements}
\ednote{J\"org,...}
\end{acknowledgements}

\printbibliography
\end{document}
%%% Local Variables:
%%% mode: latex
%%% TeX-master: t
%%% End:

%  LocalWords:  maketitle tbw sec:intro NewSim:ltmcips56 SoftwareHeritage:on sec:concl
%  LocalWords:  printbibliography
